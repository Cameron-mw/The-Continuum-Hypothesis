%==================================================================
%
% Walla Walla University Mathematics
% MATH 497 -- Seminar
% Paper Template
%
%==================================================================

\documentclass[10pt,letterpaper]{amsart}

%==================================================================
% Included Packages -- feel free to add to this list!
\usepackage{amsmath}
\usepackage{amssymb}
\usepackage{amsthm}
\usepackage{fancyhdr}
\usepackage{graphicx}

%==================================================================
% Margin Definitions -- these should be left alone is most cases

% text dimensions
\setlength{\textwidth}{6.5in}
\setlength{\textheight}{9in}

% adjust top margins
\setlength{\topmargin}{0in}
\setlength{\voffset}{-30pt}

% no room for notes on the side
\setlength{\oddsidemargin}{0in}
\setlength{\evensidemargin}{0in}
\setlength{\marginparwidth}{0in}

% space between paragraphs and lines
\setlength{\parskip}{5pt} 
\linespread{1.05}

%==================================================================
% Hyphenation

\numberwithin{equation}{section}
\hyphenation{semi-stable}

%==================================================================
% Numbering for Theorems, Lemmas, etc.

\theoremstyle{plain}
\newtheorem{theorem}{Theorem}[section]
\newtheorem{corollary}[theorem]{Corollary}
\newtheorem{lemma}[theorem]{Lemma}
\newtheorem{conjecture}[theorem]{Conjecture}

\theoremstyle{definition}
\newtheorem{definition}[theorem]{Definition}
\newtheorem{example}[theorem]{Example}
\newtheorem{remark}[theorem]{Remark}

\numberwithin{equation}{section}

%==================================================================
% Header Definitions

\fancyhead[L]{\nouppercase{\rightmark}}
\fancyhead[R]{\nouppercase{\leftmark}}


%==================================================================
% Star of the document

\begin{document}

\pagestyle{plain}

%==================================================================
% Title Information

\title{The Continuum Hypothesis and Interpretations in the Philosophy of Mathematics}
\author{Cameron M. Woodward}
%\address{Department of Mathematics, Walla Walla University, College Place, WA 99324}

\begin{abstract}
The Continuum Hypothesis (CH) was the first of 23 unsolved problems collected and proposed by David Hilbert to be the most important and influential problems of mathematics in the 20th century. The Continuum Hypothesis is a simple statement that there does not exist a set whose cardinality is between that of the natural numbers and the real numbers. This question of the continuum has existed since Georg Cantor formalized many of the ideas of set theory used today, and the intrigue surrounding the hypothesis has continued to this day as well. A formal proof eluded mathematicians until Kurt Gödel's and Paul Cohen’s proofs in the mid 20th century collectively showed that the Continuum Hypothesis is independent of ZFC set theory, meaning that ZFC is consistent with CH as an axiom or $\neg$CH as an axiom. While the independence of CH is firmly established, further investigation into CH's relationship with set theory is ongoing today.
\end{abstract}

\maketitle

%==================================================================
% First Section -- usually an introduction

\section{Introduction}
%\subsection{}

Mathematics is a process of great precision when used to solve problems in the applied sciences, and an equal level of precision is involved in solving problems in subjects of `non-applied' or pure mathematics such as real analysis. Real analysis deals with some of the underlying set-theoretic proofs that motivate the tools of calculus, which are foundational to physics, an applied science. The fields of applied and pure maths are not mutually exclusive, and while set theory was established after many mathematical tools were already in use such as trigonometry and physics, set theory is constructed to be an explicit mathematical foundation laying far beneath many of these older practices. The discussion of the Continuum Hypothesis (CH) lays firmly in the realm of pure mathematics, and raises questions in philosophy as well, but should not be thought of as separate from the mathematical practices of everyday. CH questions the size and limits of the real number line, something that most students will encounter at some point in their academic careers. \\

To give context to the continuum problem, we will begin with the discussion of size, and what it means for two sets to have the same size. This will bring us to talk about axioms: both what they are and how they operate in set theory. CH can't be fully understood without the concepts set size and set membership, and the proofs showing independence can not be comprehended without some mention of the axioms of our modern day set theory. Furthermore, in order to navigate these proofs, there must be an understanding of the definitions of independence, the differences between models and theories, soundness and sufficiency, and so on. Once the proofs of Kurt Gödel and Paul Cohen are discussed, we can talk about their interpretations and the status of CH in set theory today.


%==================================================================
% Next Section -- now we get into the content

\section{Body}

Georg Cantor introduced our modern conception of cardinality when he described the equivalence of two sets by the existence of a one-to-one correspondence between them (Abbott, 36-37). This definition of cardinality is foundational to modern set theory.\\

\noindent Definition 1:
The set $A$ has the same cardinality as set $B$ if there exists $f:A\rightarrow B$ that is 1-1 and onto (Abbott, 25). The cardinality of set $A$ is written as $|A|$.\\

The cardinality of the natural numbers (or counting numbers) $\mathbb{N}$ is denoted by $\aleph_0$ (aleph-null). This is the smallest infinite number, also referred to as the first transfinite cardinal. The set of naturals, along with the sets of integers, rationals, algebraic, and constructible numbers are all $\aleph_0$ sized sets and are countably infinite (Abbott, 25). \\

\noindent Definition 2:
A set $A$ is countably infinite if and only if $|A|=\aleph_0$. \\

For undergraduates, Cantor is most famous for the diagonalization argument he gave proving that the cardinality of the set of real numbers $|\mathbb{R}|$ is strictly greater that the cardinality of the natural numbers $|\mathbb{N}|$ (Abbott, 32-33). Cantor's diagonalization method can be found in most real analysis texts, and its implications span across set theory. Cantor's proof provide the following results.\\

\noindent Theorem: Cardinal arithmetic. \\
For any cardinal $\alpha$, $2^\alpha>\alpha$, and $2^\alpha \geq \alpha^+$, where $\alpha^+$ denotes the successor of $\alpha$. \\

\noindent The Axiom of the Power Set: \\
``For any $X$ there exists a set $Y=P(X)$, the set of all subsets of $X$" (Weisstein). Formally, \\ $\forall X \exists Y \left[ \forall u(u\in Y \iff u \subseteq X) \right] $. 
Note that the cardinality of a set $A$ is strictly less than the cardinality of its power set $|P(A)|$, where $|P(A)|=2^{|A|}$. \\

The question of the Continuum Hypothesis (CH) arises when we invoke the power set operator on sets of infinite cardinality. The most concise symbolic form of CH is $2^{\aleph_0}=\aleph_1$, where $2^{\aleph_0}$ is the cardinality of the power set of the natural numbers. What is $\aleph_1$? There is no dispute that this number exists and that it is surely the successor of $\aleph_0$. In fact, it is a member of the aleph sequence, the sequence of infinite cardinals of which $\aleph_0$ is the first: $\{ \aleph_0, \aleph_1, \aleph_2, ..., \aleph_\omega, ...\}$, where $\omega$ is the first infinite ordinal (R. Benton, Personal Communication). The questions posed and definitively answered by CH are these: Is $2^{\aleph_0}$ the successor to $\aleph_0$, and is  $\aleph_1$ equal to $2^{\aleph_0}$? \\

\noindent Hypothesis: CH. \\
There is no set whose cardinality is strictly between that of the integers and the real numbers.\\

The Continuum Hypothesis is the statement that there does not exist a set whose cardinality is between $\aleph_0$ and $\aleph_1$, and precisely that $2^{\aleph_0}=\aleph_1$. In fact, CH is just a specific and special case of the broader Generalized Continuum Hypothesis (GCH), similarly expressed by $2^{\aleph_\alpha}=\aleph_{\alpha+1}$ for all ordinals $\alpha$. GCH naturally demonstrates that the use of the power set operation moves us up the aleph sequence. It should be noted that in conjunction with the aleph sequence is the beth sequence $\{ \beth_1, \beth_2, \beth_3,...\}$, where each beth number represents the cardinality of the power set of its predecessor, given by $2^{\beth_\alpha}=\beth_{\alpha+1}$ for ordinals $\alpha$ (R. Benton, Personal Communication). CH is a statement about the size of infinite numbers, and consequently posits a mapping between these discrete aleph and beth sequences where $(\aleph_0=\beth_0, 2^{\beth_0}=\beth_1,...)$.  CH states that no set's cardinality exists `between' that of the natural numbers and the real numbers, and this size rule should follow between all sets higher up in the sequence as well. So if a set is larger than the naturals, then it is at least as large as the reals. As the name suggests, CH is a hypothesis, and has not been proven to be true per se. The precise truth value of CH however, is much more complex. To fully understand the nature of its validity or lack thereof, we must first be aware of some general terms in set theory.\\

Set theory is a foundational subject in the field of mathematics as all contemporary proofs done in real analysis or abstract algebra for example are expressed in the language of sets. The mainstream set theory we use today is called Zermelo-Fraenkel Set Theory with the Axiom of Choice (ZFC or ZF without choice) (Weisstein). \\

\noindent Definition 3: 
An axiom of set theory is a statement (or sentence) that is widely regarded or agreed upon as true or irrefutable, and is used to prove other statements or theorems within the set theory. \\

A set theory $T$ is consistent if any proofs invoking its axioms (or theorems) do not lead to contradictions, where a contradiction generally is a result where a statement $p$ and its negation $\neg p$ both follow proof-wise from $T$-axioms ($T\vdash p$ and $T\vdash \neg p$) (R. Benton, Personal Communication). Unfortunately, Gödel's Incompleteness Theorems demonstrate that any system of axioms sufficient to do higher mathematics cannot prove its own consistency and that there will always be unprovable but true statements within the system. In practice, ZFC has established a useful and credible foundation of mathematics by its applicability to empirical science (e.g. physics), and its results in pure mathematics (Colyvan, 1). A ZF axiom may be slightly restricted or refined from time to time, but most mathematicians today do not worry that an existential threat to ZFC is lurking in the unknown. Nevertheless, Gödel's Incompleteness Theorem entirely prevents us from proving the consistency of ZF axioms using themselves, and that syntactical limits exist in every set theory (R. Benton, Personal Communication). Note that syntax (consistency) is not identical to truth (semantics).\\

When discussing CH, we use terms such as theory, model, and consistency. It is important to understand precisely what these terms mean in this context. We will talk about models, structures in which sentences, or sets of sentences called theories can be tested and found to be true or false (R. Benton, Personal Communication). These models are best thought of as collections of sets (e.g. groups, rings, $\mathbb{R}, \mathbb{N}$, etc.). The expression $T\models A$ means that the property $A$ is true in all models of the theory $T$. If all statements of a theory $T$, (such as ZFC) are true in a model $M$, then we say that $M$ is model of $T$.  \\

\noindent Definition 4: Soundness and Sufficiency. \\
1. A theory $T$ is said to be sound if and only if $T\vdash A \implies T\models A$ for some property A, and \\
2. $T$ is sufficient if and only if $T\models A \implies T\vdash A$. \\
3. The theory $T$ is said to be complete if and only if $T\vdash A \iff T\models A$ (R. Benton, Personal Communication). \\

Generally, ZFC is summarized by nine axioms, but this number may vary slightly from source to source (Weisstein). These axioms are stated very carefully, and their details are (mostly) beyond the scope of this paper except for the Axiom of Choice (AC), and the Axiom of the Power Set which we discussed previously. The Axiom of Choice is a statement about the properties of an infinite number (or family) of sets. While controversial in the 20th century, AC is widely accepted by contemporary mathematicians, and is relevant to the study of CH in two ways: by its relationship with infinite objects and its independent nature. \\

\noindent The Axiom of Choice (AC): \\
Every collection of non-empty sets has a choice function. Formally, \\
$\forall A \left[ (\forall B \in A )(B\neq \emptyset \rightarrow (\exists f:A \rightarrow \cup A) (\forall B \in A (f(B)\in B))\right] $ (R. Benton, Personal Communication). \\

A cursory example: Suppose the set of sets $A=\left\{ \{1,2,3\}, \{5,6\}, \{7,9,10\} \right \}$. If we say, ``Select the smallest number from each subset of A", and produce $\{1,5,7\}$, then we used a choice function. A choice function is defined on a collection of sets and specifies or selects a specific element of each set. We need not invoke the Axiom of Choice in this specific example, because the collection $A$ is finite. AC is only practically applied to infinitely sized families. Because there is no choice function for the collection of all non-empty subsets of the real numbers, we invoke AC which says one exists. Both AC and CH possess a remarkable property that is given only by the completion of separate proofs from Kurt Gödel in 1939 and Paul Cohen in 1963. This property will be the focus for the remainder of the discussion here, and that is independence.\\

\noindent Definition 5:
An axiom or statement $A$ is independent from our set theory (Zermelo-Fraenkel Set Theory with the Axiom of Choice) if ZFC axioms neither prove nor refute it, that is, $ZFC \nvdash \neg A$ and $ZFC \nvdash \neg(\neg A)$ (R. Benton, Personal Communication). \\

The two proofs which only together give the Axiom of Choice's independence result show that both $AC$ and $\neg AC$ are consistent with ZF set theory. Like AC, the Continuum Hypothesis is independent of ZF set theory as well, and the same follows: that $ZF+CH$ and $ZF+\neg CH$ are both consistent set theories. The usual method of showing that a certain statement $A$ is independent and that neither $A$ nor its negation $\neg A$ are derivable from given axioms is to create separate models where the axioms are true, and show $A$ to be true in one model, but false in the other. This is the type of process that is used to show CH's and AC's independence. \\

The Austrian mathematician Kurt Gödel was the first to provide a proof of one side of CH's independence. To be more precise, his proof only addresses the consistency of $ZF+GCH$ (i.e. $ZF \nvdash \neg CH$) using a constructible model of ZF (R. Benton, Personal Communication). The consistency result for $ZF+\neg GCH$ (i.e. $ZF  \nvdash \neg(\neg CH)$) was later shown by the mathematician Paul Cohen in 1963 (Koellner, 1) . Only both of these mens' results together will establish CH's independence. The technique Gödel used for the $ZF+GCH$ proof is called the the method of Inner Models, which we will discuss after establishing the definition of a proper class and the Von-Neumann Universe of Sets. \\

\noindent Definition 6: A proper class is a `large' collection of sets that can be defined by a common property of the sets contained within it, but is not a set itself. \\ 

The Von-Neumann Universe of Sets $V$ is a proper class that is constructed via the Axiom of Union by assigning a set $V_\alpha$ to each ordinal $\alpha$ by a recursively defined list of rules. The Von-Neumann Universe is defined by these steps of transfinite induction:\\
1. $V_0=\emptyset$; \\
2. $V_{\alpha+1}=P(V_\alpha)$, where $P()$ is the power set operation; \\
3. $V_\alpha=\bigcup_{\beta < \alpha} V_\beta$ when $\alpha$ is a limit ordinal (not a successor of another ordinal) (Singh et. al, 496, 497). \\

Following these steps, $V_0=\emptyset, V_1=\{ \emptyset \}$, $V_2=\{ \emptyset, \{ \emptyset \} \}$, and so on, resulting in a collection of sets built up from the empty set via transfinite induction. $V$ is referred to as the cumulative hierarchy of sets, and each level of the hierarchy can be indexed by ordinals, and copies of ordinal-behaving sets can be found inside this hierarchy (Singh et. al, 497). In the finite case, each set $V_\alpha$ contains something that behaves exactly like a cardinal as well (i.e. $\emptyset = 0, \{ \emptyset \}=1, \{ \emptyset, \{ \emptyset \} \}=2,... $). For example, if we let $n+1=\{ 0,1,2,...,n\}$, then these sets behave like natural numbers which are the finite ordinals and cardinals. Note that $\alpha \subseteq V_\alpha$, not just for finite ordinals, but for infinite ordinals as well (R. Benton, Personal Communication). While seeming to be a contrived construction at first, the Von-Neumann Universe captures the intuitive properties of the ZF axioms so that it is reasonable to believe if a statement is true in $V$, it is true in ZF as well (assuming ZF is consistent). The $V$ universe provides a familiar background for the constructible class $L$ where Gödel's proof of $ZF+GCH$ takes place.  \\

Returning to Gödel, he will find a model of ZFC in which CH is true (or $\neg CH$ is false). His proof is done using the method of inner models, a technique which `shrinks' the size of $V$ and shows consistency within the `smaller' or inner model of ZF. This inner model Gödel calls the constructible universe $L$, which contains all the ordinals in $V$ (Rosello, 174). The class $L$ will serve as a model of ZF set theory. Its construction is similar to that of $V$ in the iterative sense with the exception that the power set operation generating the $\alpha+1$ level is further restricted in the sets it can generate, and so $L_\alpha \subset V_\alpha$ for successor ordinals $\alpha$ (Rosello, 174). The following definition for $L$ comes from Joan Rosello's ``From Foundations to Philosophy of Mathematics". \\

\noindent Definition: $L$ Constructible Universe.\\
1. ``$L_0=\emptyset$; \\
2. $L_{\alpha}=P(L_\beta)$, if $\alpha=\beta+1$, where $L_\beta$ is first order definable; \\
3. $L_\alpha=\bigcup_{\beta < \alpha} L_\beta$, if $\alpha$ is a limit ordinal" (Rosello, 174). \\

The key difference between the construction steps of $L$ and $V$ is that each $L_\beta$ must be first order definable as stated in step (2) above. A language is first order definable if its axioms range over first order variables (i.e. objects such as $x$ and $y$), and its functions map objects from sets into sets. A language having functions ($A$, $B$, etc) that range over certain (more complex) properties of sets belongs to the realm of higher order logic, where paradoxical results may become more prevalent. In the context of $L$, this notion of first order definability applies uniqeuly to the sets beyond $L_\omega$ where we are taking a union over an infinite quantity of sets from the previous stage (R. Benton, Personal Communication). Not all subsets of $V_{\omega+1}$ meet this criteria of first order definability, and so, they are therefore not included in the more restricted set $L_{\omega+1}$. \\

\noindent The Axiom of Constructibility (ACon):\\
Every set of $V$ is constructible in $L$, therefore $V=L$ (Rosello, 174).\\

A set may be called `constructible' if it is not defined by an impredicative procedure, that is, a self-referencing function used to further construct sets like itself. The notion of Gödel's Constructible Universe modeling ZF is summarized by the Axiom of Constructibility and the analogous proposition $V=L$, whose interpretation we will discuss later. The truth of ACon is irrelevant to Gödel's proof of $ZF+GCH$ since $L$ is clearly constructible from ZF axioms. Gödel's 1939 method by which GCH is satisfied in $L$ now follows. ``From Foundations to Philosophy of Mathematics" by Joan Rosello summarizes it this way:\\

\noindent Lemma: ``Be $\lambda$ an infinite cardinal and $x$ an arbitrary subset of $\lambda$; If $V=L$, then $x$ is a member of $L_{\lambda^+}$" (Rosello, 175). \\

In this case, if $V=L$ is accepted, then $| L_{\lambda^+} |$ is the smallest cardinal greater than $\lambda$, and it follows that $L_{\lambda^+}$ has the same cardinality as $\lambda^+$ and $L$ satisfies GCH (Rosello, 175). \\

Gödel would later speculate that GCH was not only consistent in a model of ZFC as he had shown, but was in fact independent of ZFC axioms as we now know. Its independence went from speculation to fact in 1963 when American mathematician Paul Cohen constructed a model of ZF in which CH is false (or $\neg$CH is true). Cohen achieved this by the method of forcing, whose concepts we will now summarize. Whereas Gödel's inner model of ZF was a class from $V$ minimized so much that it barely could barely contain the ZF axioms, Cohen's `outer model' can be thought of as the opposite. Forcing uses the Cartesian product to add new elements to existing sets in $V$ to create conditions where $\neg CH$ is satisfied. ``A beginner's guide to forcing" by Timothy Y. Chow summarizes the construction of Cohen's model at a conceptual level that is desirable here. \\

The goal in Cohen's method is to define a specific function $F$ that can be joined to the model $M$ in such a way that $M$'s properties are maintained, that is, $M$ remains a ``standard transitive model" of ZFC (Chow, 7). The larger model created would thus be called $M[F]$. It is self evident that if $M$ is a model of ZFC, there exists a set in $M$ of equivalent cardinality to $\aleph_2$ in ZFC. This set is denoted as $\aleph_2^M$ (Chow, 7). Now, define $F$ to be the Cartesian product of ``$\aleph_2^M$ x $\aleph_0$ into the set $2=\{0,1\}$", so that $F$ is a sequence of functions from $\aleph_0$ into 2 (Chow, 7). Since $M$ is countable and transitive, then $\aleph_2^M$ is as well, and they can be arranged in a way so that they are pairwise distinct (Chow, 7). Functions from $\aleph_0$ are associated with subsets of $\aleph_0$, and so the power set of $\aleph_0$ in $M$ must be at least the size of $\aleph_2$ in M, therefore $M$ satisfies $\neg CH$ (Chow, 7). The trickiest part of this method of this forcing is not showing the end result itself, but rather verifying that $F$ does not change $M$ in such a way that it no longer models ZFC. By adding the function $F$  to the model, we are necessarily also adding every set that can be constructed from $F$ with elements of $M$ (Chow, 7). Both Chow and Cohen go into details far beyond our naïve level here about the technicalities and difficulties inherent so such a broad brushed method such as forcing, and reading their insights is encouraged. \\

A more intuitive yet equally rigorous approach to the proof of $ZF+\neg CH$ is called the Boolean Valued Model which was developed by Dana Scott shortly after Cohen published his work. This process is demonstrated in Scott's 1967 paper titled ``A Proof of Independence of the Continuum Hypothesis".\\


%==================================================================
% Another section, with references to previous entities

\section{Conclusion and Interpretations}

Gödel and Cohen leave distinguished legacies among 20th century set theorists by their work and it seems their proofs about GCH had philosophical impacts on themselves as mathematicians as they do on us today. In this conclusion we shall tap into the questions of how their beliefs were shaped by their work, how we reconcile unprovable statements, and what the status is of the Continuum Hypothesis in modern set theory. \\

Gödel's philosophical stance is commonly associated with Platonism.\;Mathematical Platonism simply stated is the belief that mathematical objects exist in our universe and are independent of human consciousness (Linnebo, 1). Abstract constructions in mathematics such as sets or the concept of number itself while not being physical manifestations in nature nevertheless possess an ontological status according to Platonism. A Platonist like Gödel would believe that mathematical principles `exist' since they are products of our intuition and that impredicative definitions are legitimate (Linnebo, 1). In the Stanford Encyclopedia of Philosophy's article ``Philosophy of Mathematics", Leon Horsten states that for Gödel, 
\begin{quote}
``Our mathematical intuition provides intrinsic evidence for mathematical principles. Virtually all of our mathematical knowledge can be deduced from the axioms of Zermelo-Fraenkel set theory with the Axiom of Choice (ZFC). In Gödel’s view, we have compelling intrinsic evidence for the truth of these axioms. But he also worried that mathematical intuition might not be strong enough to provide compelling evidence for axioms that significantly exceed the strength of ZFC" (Horsten, 3.1).
\end{quote}


Gödel reconciled the independence result of GCH with non-Pluralism.\;The general idea of the non-Pluralist movement is that independence results such as GCH's merely show that our current set theories are inadequate and that we need new and more powerful axioms to settle CH (Koellner, 1). After the independence result, the study of so called `large cardinal axioms' became relevant: axioms which produce powerful results by equally powerful assumptions about the properties of large cardinal numbers in higher set theory. It has become apparent that these kinds of axioms must be considered in the current study of CH (Koellner, 1). As far as ACon is concerned, its discussion and its non-acceptance as a ZFC axiom is both ongoing and complex. Among skeptics, it seems that ACon's strictly constructivist nature may be too restrictive on what kinds of sets can exist, and that the limitations placed on the power set operation may be too arbitrary in the sets it allows to exist without clear enough reasons (Eskew, 2019). Ultimately, $V=L$ is an undecidable sentence, and after Cohen's work established CH's independence, Gödel would come to believe that further investigations into CH and will require large cardinal axioms and their properties (Koellner, 1) (Rosello, 175). \\

Cohen earlier in his career subscribed to the Formalist philosophy of David Hilbert, whose summary idea is that mathematics itself is a formal game, and that proofs are merely the manipulations of symbols according to a set of fixed rules (Horsten, 2.4). Formalism is contrasted to Platonism in that formalists may reject abstract principles outside the results of manipulations of set mathematical rules. Formalism by and large lost its momentum after Gödel's Incompleteness Theorems were published. Opposite to Gödel, Cohen moved to the Pluralist school of thought, and his reconciliation of the independence of GCH is that there is no answer: that both CH and $\neg$CH are equally useful and valid (Koellner, 1).\\

The notion of independence itself will be unsettling to some. Gödel and Cohen both took philosophical viewpoints in order to justify what their work together had established. Is CH's independence satisfying? Can something exist in the world as both true and false? Gödel's non-Pluralist reconciliation is a more restless reflection to these problems, whose solution is that our current theories are not good enough, and that the truth `exists' somewhere we are yet to find (truth-value realism) (Linnebo, 1.4). Cohen's Pluralist response to independence is more utilitarian in that CH's truth value can be altered in order to make ZFC a more powerful theory in addressing a particular problem. Cohen's response may be considered more optimistic, but to say so might be an oversimplification. \\

The Continuum Hypothesis remains a popular topic in set theory today. Despite the clear independence results given by Gödel and Cohen, its study is still ongoing, and CH remains a famous mathematical and philosophical question. Some contemporary set theorists are attempting to resolve a definitive truth value of CH through numerous kinds of higher order logic and large cardinal axioms (e.g. Woodin Cardinals, $\Omega$-Logic, etc.) (Koellner, 3.2). As stated previously, the consistency of ZFC with large cardinal axioms has not been shown. It is possible that certain axioms of large cardinal properties may be added to ZFC in the future, but this is speculation. Regardless, the Continuum Hypothesis continues to challenge the mathematical conceptions of size, order, and membership. Problems like CH test the limits and the power of our set theories, and continuously motivate us in the search for truth.\\

\begin{thebibliography}{99}

\bibitem{} Abbot, Stephen. Understanding Analysis. Second Edition, Springer, 2015. \\
  
\bibitem{} Benacerraf, Paul, and Hilary Putnam, editors. Philosophy of Mathematics: Selected Readings. 2nd ed, Cambridge University Press, 1983.\\

\bibitem{} Chow, Timothy Y. “A Beginner’s Guide to Forcing.” Contemporary Mathematics, edited by Timothy Y. Chow and Daniel C. Isaksen, vol. 479, American Mathematical Society, 2009, pp. 25–40. DOI.org (Crossref), https://doi.org/10.1090/conm/479/09340.\\

\bibitem{} Cohen, Paul J. The Independence of the Continuum Hypothesis I. 2022, p. 7.\\

\bibitem{} Eskew, Monroe. “Why Not Adopt the Constructibility Axiom $V=L$?” MathOverflow, https://mathoverflow.net/q/331956.\\

\bibitem{} Gödel, Kurt. The Consistency of the Axiom of Choice and the Generalized Continuum Hypothesis. p. 2.\\

\bibitem{} Roselló, Joan. From Foundations to Philosophy of Mathematics: An Historical Account of Their Development in the XX Century and Beyond. Cambridge Scholars, 2012.\\

\bibitem{} Singh, D., and J. N. Singh. “Von Neumann Universe: A Perspective.” International Journal of Contemporary Mathematical Sciences, vol. 2, 2007, pp. 475–78. DOI.org (Crossref), https://doi.org/10.12988/ijcms.2007.07043.\\

\bibitem{} Weisstein,\;Eric\;W.\;Zermelo-Fraenkel Axioms.\;Mathworld--A Wolfram\;Web Resource,\;2019,\\https://mathworld.wolfram.com/Zermelo-FraenkelAxioms.html.\\

\bibitem{} Koellner, Peter, ``The Continuum Hypothesis", The Stanford Encyclopedia of Philosophy (Spring 2019 Edition), Edward N. Zalta (ed.), URL = https://plato.stanford.edu/archives/spr2019/entries/continuum-hypothesis/. \\

\bibitem{} Horsten, Leon, ``Philosophy of Mathematics", The Stanford Encyclopedia of Philosophy (Spring 2022 Edition), Edward N. Zalta (ed.), URL = https://plato.stanford.edu/archives/spr2022/entries/philosophy-mathematics/. \\

\bibitem{} Linnebo, Øystein, ``Platonism in the Philosophy of Mathematics", The Stanford Encyclopedia of Philosophy (Spring 2018 Edition), Edward N. Zalta (ed.), URL = https://plato.stanford.edu/archives/spr2018/entries/platonism-mathematics/. \\

\bibitem{} Colyvan, Mark, ``Indispensability Arguments in the Philosophy of Mathematics", The Stanford Encyclopedia of Philosophy (Spring 2019 Edition), Edward N. Zalta (ed.), URL = https://plato.stanford.edu/archives/spr2019/entries/mathphil-indis/. \\

\end{thebibliography}

\end{document}
